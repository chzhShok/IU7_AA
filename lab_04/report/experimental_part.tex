\chapter{Исследовательская часть}

\section{Характеристики ЭВМ}
Замеры проводились на устройстве со следующими характеристиками:

\begin{itemize} 
	\item процессор: Apple M4 Pro;
	\item количество логических ядер: 12;
	\item количество ядер: 12;
	\item оперативная память: 24 Гб;
	\item операционная система: macOS Sequoia 15.6.1.
\end{itemize}

Замеры времени проводились, когда ноутбук был загружен только системными приложениями.

\section{Замер времени}

В данном разделе представлены результаты измерения времени выполнения параллельного алгоритма обработки графов. 

Замеры проводились для графов размером от 3000 до 10000 вершин. Для каждого размера графа выполнялись измерения времени выполнения при различном количестве рабочих потоков: 0 (последовательная версия), 1, 2, 4, 8, 12, 16, 32, 64. Каждый замер выполнялся 3 раза для получения статистически значимых результатов.

\begin{table}[h!]
	\centering
	\small
	\caption{Результаты измерения времени выполнения, часть 1 (мкс)}
	\begin{tabular}{|c|r|r|r|r|r|}
		\hline
		\textbf{Вершин} & \textbf{0 поток.} & \textbf{1 поток} & \textbf{2 поток.} & \textbf{4 поток.} & \textbf{8 поток.} \\
		\hline
		3000 & 5648 & 1231 & 1909 & 2002 & 2940 \\
		\hline
		4000 & 10192 & 1679 & 2468 & 2777 & 3624 \\
		\hline
		5000 & 15904 & 2218 & 3322 & 3558 & 5881 \\
		\hline
		6000 & 22362 & 2733 & 4060 & 4259 & 5986 \\
		\hline
		7000 & 29740 & 3185 & 4459 & 4535 & 7806 \\
		\hline
		8000 & 38047 & 3511 & 5066 & 5194 & 8886 \\
		\hline
		9000 & 48894 & 4048 & 5766 & 5915 & 10046 \\
		\hline
		10000 & 62163 & 4691 & 6522 & 6487 & 11102 \\
		\hline
	\end{tabular}
	\label{table:time_measurements_1}
\end{table}

\begin{table}[h!]
	\centering
	\small
	\caption{Результаты измерения времени выполнения, часть 2 (мкс)}
	\begin{tabular}{|c|r|r|r|r|}
		\hline
		\textbf{Вершин} & \textbf{12 поток.} & \textbf{16 поток.} & \textbf{32 поток.} & \textbf{64 поток.} \\
		\hline
		3000 & 2883 & 2939 & 3419 & 3969 \\
		\hline
		4000 & 4129 & 3800 & 4079 & 4641 \\
		\hline
		5000 & 5697 & 5773 & 5318 & 5706 \\
		\hline
		6000 & 7016 & 6687 & 6261 & 6436 \\
		\hline
		7000 & 8430 & 8172 & 7219 & 7301 \\
		\hline
		8000 & 9446 & 9079 & 8201 & 8390 \\
		\hline
		9000 & 10862 & 10513 & 9312 & 9540 \\
		\hline
		10000 & 23138 & 11683 & 10302 & 10626 \\
		\hline
	\end{tabular}
	\label{table:time_measurements_2}
\end{table}

\begin{figure}[h!]
	\center{\includegraphics[scale=0.55]{./images/time_comparison.png}} 
	\caption{Сравнение времени выполнения при различном количестве потоков}
	\label{fig:time_comparison}
\end{figure}

\vspace{100pt}

\section{Анализ масштабируемости}

На основе полученных результатов был проведён анализ масштабируемости параллельного алгоритма. Был рассчитан показатель ускорения для различных конфигураций.

\begin{table}[h!]
	\centering
	\small
	\caption{Ускорение параллельного алгоритма, часть 1}
	\begin{tabular}{|c|r|r|r|r|}
		\hline
		\textbf{Вершин} & \textbf{1 поток} & \textbf{2 потока} & \textbf{4 потока} & \textbf{8 потоков} \\
		\hline
		3000 & 4.59x & 2.96x & 2.82x & 1.92x \\
		\hline
		4000 & 6.07x & 4.13x & 3.67x & 2.81x \\
		\hline
		5000 & 7.17x & 4.79x & 4.47x & 2.70x \\
		\hline
		6000 & 8.18x & 5.51x & 5.25x & 3.74x \\
		\hline
		7000 & 9.34x & 6.67x & 6.56x & 3.81x \\
		\hline
		8000 & 10.84x & 7.51x & 7.33x & 4.28x \\
		\hline
		9000 & 12.08x & 8.48x & 8.27x & 4.87x \\
		\hline
		10000 & 13.25x & 9.53x & 9.58x & 5.60x \\
		\hline
	\end{tabular}
	\label{table:scalability_1}
\end{table}

\begin{table}[h!]
	\centering
	\small
	\caption{Ускорение параллельного алгоритма, часть 2}
	\begin{tabular}{|c|r|r|r|r|}
		\hline
		\textbf{Вершин} & \textbf{12 потоков} & \textbf{16 потоков} & \textbf{32 потока} & \textbf{64 потока} \\
		\hline
		3000 & 1.96x & 1.92x & 1.65x & 1.42x \\
		\hline
		4000 & 2.47x & 2.68x & 2.50x & 2.20x \\
		\hline
		5000 & 2.79x & 2.75x & 2.99x & 2.79x \\
		\hline
		6000 & 3.19x & 3.34x & 3.57x & 3.47x \\
		\hline
		7000 & 3.53x & 3.64x & 4.12x & 4.07x \\
		\hline
		8000 & 4.03x & 4.19x & 4.64x & 4.53x \\
		\hline
		9000 & 4.50x & 4.65x & 5.25x & 5.13x \\
		\hline
		10000 & 2.69x & 5.32x & 6.03x & 5.85x \\
		\hline
	\end{tabular}
	\label{table:scalability_2}
\end{table}

\begin{figure}[h!]
	\center{\includegraphics[scale=0.6]{./images/scalability_analysis.png}} 
	\caption{Анализ ускорения параллельного алгоритма}
	\label{fig:scalability}
\end{figure}

\section{Сравнительный анализ}

\subsection{Сравнение последовательного и параллельного алгоритмов}

При сравнении последовательного алгоритма (0 потоков) с параллельной версией, использующей один рабочий поток, наблюдается значительное ускорение от 4.59x до 13.25x в зависимости от размера графа. Это свидетельствует об эффективности разработанного параллельного подхода даже при минимальном количестве потоков.

\begin{figure}[h!]
	\center{\includegraphics[scale=0.5]{./images/sequential_vs_parallel.png}} 
	\caption{Сравнение последовательного и параллельного алгоритмов}
	\label{fig:0_vs_1}
\end{figure}

\subsection{Влияние количества потоков на производительность}

Анализ результатов демонстрирует значительное ускорение параллельной версии по сравнению с последовательной:

\begin{itemize}
	\item максимальное ускорение достигается при использовании 1 потока для всех размеров графов;
	\item для небольших графов (3000-4000 вершин) ускорение составляет 4.59-6.07x;
	\item для крупных графов (8000-10000 вершин) ускорение достигает 10.84-13.25x.
\end{itemize}

\section*{Вывод}

Выводы на основании проведённых измерений:

\begin{enumerate}
	\item разработанный параллельный алгоритм демонстрирует значительное ускорение по сравнению с последовательной версией -- от 4.59x до 13.25x в зависимости от размера графа;
	\item наилучшая производительность достигается при использовании 1 рабочего потока для всех исследованных размеров графов (3000-10000 вершин);
	\item с увеличением количества потоков наблюдается тенденция с увеличением времени работы алгоритма.
\end{enumerate}

\clearpage
