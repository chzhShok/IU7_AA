\chapter{Конструкторская часть}

\section{Требования к реализации}

К программе предъявлены следующие функциональные требования.

\vspace{10pt}

\textbf{Входные данные}
\begin{itemize}[label=---]
	\item граф в формате DOT, содержащий ориентированный взвешенный граф;
	\item стартовая вершина -- имя вершины, от которой ищутся пути;
	\item целевые вершины -- список имён вершин, до которых ищутся пути;
	\item количество потоков -- целое число (0 для последовательного режима, > 0 для параллельного).
\end{itemize}

\vspace{10pt}

\textbf{Выходные данные}
\begin{itemize}[label=---]
\item имя стартовой вершины;
\item массив имён целевых вершин;
\item количество используемых потоков;
\item режим работы (последовательный или параллельный);
\item время выполнения алгоритма в миллисекундах;
\item объект с расстояниями до каждой целевой вершины;
\item объект с информацией о кратчайшем пути, содержащий: имя целевой вершины с минимальным расстоянием, длину кратчайшего пути, массив имён вершин, составляющих путь от стартовой до целевой вершины.
\end{itemize}

\vspace{10pt}

\textbf{Функциональные требования}
\begin{itemize}[label=---]
	\item поддержка загрузки графов из файлов формата DOT;
	\item реализация последовательного алгоритма Дейкстры;
	\item реализация параллельного алгоритма Дейкстры с использованием нативных потоков;
	\item обработка некорректных входных данных с выводом сообщений об ошибках;
	\item вывод результатов в формате JSON для последующей обработки;
	\item замер времени выполнения алгоритмов.
\end{itemize}

\vspace{10pt}

\textbf{Режимы работы} 
\begin{itemize}[label=---]
	\item последовательный режим -- выполнение алгоритма Дейкстры в одном потоке;
	\item параллельный режим -- выполнение алгоритма Дейкстры с использованием указанного количества потоков.
\end{itemize}

\section {Разработка алгоритмов}

Раздел содержит схемы алгоритмов, описывающие следующие алгоритмы: последовательный алгоритм Дейкстры~\ref{fig:seq_alg}, алгоритм работы главного потока~\ref{fig:main_thread_1},~\ref{fig:main_thread_2}, алгоритм работы вспомогательного потока~\ref{fig:dop_thread_1},~\ref{fig:dop_thread_2},~\ref{fig:dop_thread_3}.

\begin{figure}[h!]
	\center{\includegraphics[scale=0.45]{./images/seq_alg.png}} 
	\caption{Схема последовательного алгоритма Дейкстры}
	\label{fig:seq_alg} 
\end{figure}\clearpage

\begin{figure}[h!]
	\center{\includegraphics[scale=0.44]{./images/main_thread_1.png}} 
	\caption{Схема главного потока параллельного алгоритма Дейкстры, часть 1}
	\label{fig:main_thread_1} 
\end{figure}\clearpage

\begin{figure}[h!]
	\center{\includegraphics[scale=0.4]{./images/main_thread_2.png}} 
	\caption{Схема главного потока параллельного алгоритма Дейкстры, часть 2}
	\label{fig:main_thread_2} 
\end{figure}\clearpage

\begin{figure}[h!]
	\center{\includegraphics[scale=0.33]{./images/dop_thread_1.jpeg}} 
	\caption{Схема вспомогательного потока параллельного алгоритма Дейкстры, часть 1}
	\label{fig:dop_thread_1} 
\end{figure}\clearpage

\begin{figure}[h!]
	\center{\includegraphics[scale=0.5]{./images/dop_thread_2.jpeg}} 
	\caption{Схема вспомогательного потока параллельного алгоритма Дейкстры, часть 2}
	\label{fig:dop_thread_2} 
\end{figure}\clearpage

\begin{figure}[h!]
	\center{\includegraphics[scale=0.45]{./images/dop_thread_3.jpeg}} 
	\caption{Схема вспомогательного потока параллельного алгоритма Дейкстры, часть 3}
	\label{fig:dop_thread_3} 
\end{figure}\clearpage

\section*{Вывод}

В данном разделе были разработаны два алгоритма алгоритма Дейкстры: последовательный и параллельный. Для каждого из них представлены схемы алгоритмов, описывающие логику работы.

\clearpage
