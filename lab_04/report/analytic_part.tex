\chapter{Аналитическая часть}

\section{Постановка задачи}

В рамках данной работы рассматривается задача поиска кратчайших путей в ориентированном взвешенном графе~\cite{cormen,even_graphs}. Пусть задан граф $G = (V, E)$, где:
\begin{itemize}
	\item $V$ -- множество вершин графа;
	\item $E$ -- множество дуг графа, каждая дуга имеет неотрицательный вес $w(e) \geqslant 0$;
	\item $s \in V$ -- стартовая вершина;
	\item $T \subset V$ -- множество целевых вершин.
\end{itemize}

Требуется найти длины кратчайших путей от $s$ до каждой вершины $t \in T$ и определить путь минимальной длины среди них.

\section{Алгоритм Дейкстры}

\subsection{Основные положения}

\textbf{Алгоритм Дейкстры}~\cite{cormen,even_graphs} -- алгоритм нахождения кратчайшего пути от одной вершины до всех остальных в взвешенном графе с неотрицательными весами рёбер.

\vspace{10pt}

\textbf{Основные характеристики алгоритма}:
\begin{itemize}
	\item на вход -- граф $G = (V, E)$ с весами $w: E \rightarrow \mathbb{R}_{\geqslant 0}$, начальная вершина $s \in V$;
	\item на выход -- расстояния $d[v]$ от $s$ до всех $v \in V$, предки $p[v]$ для восстановления путей.
\end{itemize}

\section{Параллельный алгоритм Дейкстры}

\subsection{Общая схема распараллеливания}

Для распараллеливания алгоритма Дейкстры используется подход с \textbf{множественными очередями} (multi-queue)~\cite{quinn_parallel,williams_concurrency}. Основные идеи:

\begin{itemize}
	\item создаётся несколько очередей с приоритетами;
	\item каждый поток работает со своей очередью, но может забирать задачи из других очередей;
	\item используется work-stealing для балансировки нагрузки.
\end{itemize}

\subsection{Алгоритм работы потоков}

Каждый поток-работник выполняет~\cite{williams_concurrency,complexity}:
\begin{enumerate}
	\item пытается извлечь задачу из случайной очереди;
	\item если очередь пуста -- ожидает на условной переменной;
	\item обрабатывает извлечённую вершину;
	\item для каждого соседа обновляет расстояние атомарной операцией compare-and-swap;
	\item при успешном обновлении добавляет новую задачу в случайную очередь.
\end{enumerate}

\section{Средства синхронизации}

\subsection{Необходимость синхронизации}

В параллельном алгоритме требуются следующие средства синхронизации~\cite{williams_concurrency,quinn_parallel}:

\begin{enumerate}
	\item \textbf{мьютексы} -- для защиты доступа к отдельным очередям;
	\item \textbf{атомарные операции} -- для безопасного обновления расстояний;
	\item \textbf{условные переменные} -- для уведомления потоков о появлении новых задач;
	\item \textbf{атомарные флаги} -- для сигнализации о завершении работы.
\end{enumerate}

\subsection{Обоснование выбора}

Выбор конкретных средств синхронизации обусловлен следующими соображениями~\cite{williams_concurrency,complexity}:

\begin{enumerate}
	\item \textbf{мьютексы на каждую очередь} -- уменьшают contention по сравнению с одним глобальным мьютексом, так как потоки могут одновременно работать с разными очередями;
	\item \textbf{compare-and-swap для расстояний} -- позволяет избежать блокировок при обновлении, обеспечивая неблокирующую синхронизацию;
	\item \textbf{условные переменные} -- обеспечивают эффективное ожидание потоков без активного опроса, что снижает нагрузку на процессор;
	\item \textbf{атомарные флаги для завершения} -- позволяют безопасно координировать завершение работы всех потоков без race conditions.
\end{enumerate}

\section*{Вывод}

В аналитической части:
\begin{itemize}
	\item формализована задача поиска кратчайших путей в ориентированном взвешенном графе;
	\item описан последовательный алгоритм Дейкстры;
	\item описаны основные положения параллельного алгоритма;
	\item обоснована необходимость использования примитивов синхронизации.
\end{itemize}

\clearpage
