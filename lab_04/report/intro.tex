\ssr{ВВЕДЕНИЕ}

Целью данной работы является разработка и сравнительный анализ последовательного и параллельного алгоритмов поиска кратчайших путей в графе от стартовой вершины до нескольких конечных вершин.

\vspace{10pt}

Для достижения поставленной цели необходимо решить следующие задачи:

\begin{itemize}
	\item описать базовый последовательный алгоритм решения задачи поиска кратчайших путей в графе;
	\item разработать параллельную версию алгоритма с использованием нативных потоков;
	\item реализовать обе версии алгоритма на выбранном языке программирования;
	\item выполнить тестирование реализации алгоритмов на корректность работы;
	\item провести сравнительный анализ времени выполнения последовательного и параллельного алгоритмов;
	\item исследовать зависимость времени выполнения от количества вспомогательных потоков;
	\item сформулировать рекомендации по выбору оптимального количества потоков для данной архитектуры ЭВМ.
\end{itemize}

\vspace{10pt}

В рамках работы рассматривается граф с неотрицательными весами дуг, где заданы стартовая вершина и набор конечных вершин. Требуется найти длины путей от стартовой вершины до всех конечных и определить путь минимальной длины.

\clearpage