\ssr{ЗАКЛЮЧЕНИЕ}

В рамках данной лабораторной работы была успешно достигнута основная цель.

Все поставленные задачи выполнены в полном объёме:
\begin{itemize}[label=---]
	\item формализована задача поиска кратчайших путей в ориентированном взвешенном графе с неотрицательными весами;
	\item описан и реализован последовательный алгоритм Дейкстры;
	\item разработан параллельный алгоритм Дейкстры;
	\item проведено функциональное тестирование, подтвердившее корректность работы обоих алгоритмов на различных типах графов;
	\item выполнены замеры времени выполнения и анализ ускорения параллельного алгоритма в зависимости от количества потоков;
	\item проведён сравнительный анализ времени выполнения последовательного и параллельного алгоритмов, который показал значительное ускорение параллельной версии -- от 4.59x до 13.25x в зависимости от размера графа;
	\item исследована зависимость времени выполнения от количества вспомогательных потоков, установлено, что наилучшая производительность достигается при использовании одного рабочего потока;
	\item сформулированы рекомендации по выбору оптимального количества потоков для архитектуры ЭВМ с процессором Apple M4 Pro: для графов размером 3000-10000 вершин рекомендуется использовать 1 рабочий поток, так как увеличение количества потоков приводит к снижению производительности.
\end{itemize}

\clearpage
