\chapter{Технологическая часть}

\section{Средства реализации}

Для реализации алгоритмов подсчёта единиц в последовательности был выбран язык программирования \texttt{Rust}, поскольку он соответствует требованиям лабораторной работы. Измерение времени выполнения алгоритмов осуществлялось с использованием структуры \texttt{Instant} из стандартной библиотеки \texttt{std::time}. Разработка программного обеспечения проводилась в интегрированной среде разработки \texttt{RustRover}.

\section{Реализация алгоритмов}

Были реализованы нерекурсивный алгоритм подсчёта единиц в последовательности (листинг 3.1), рекурсивный алгоритм подсчёта единиц в последовательности (листинг 3.2).

\begin{lstlisting}[caption={Нерекурсивный алгоритм}, label={lst:iterative}]
pub(crate) fn count_until_2(slice: &[i32]) -> usize {
	let mut count = 0;
	for &x in slice {
		if x == 2 {
			break;
		}
		if x == 1 {
			count += 1;
		}
	}
	count
}
\end{lstlisting}

\begin{lstlisting}[caption={Рекурсивный алгоритм}, label={lst:recursive}]
pub(crate) fn count_until_2(slice: &[i32]) -> usize {
	if slice.is_empty() {
		0
	} else if slice[0] == 2 {
		0
	} else {
		let this = slice[0] as usize;
		this + count_until_2(&slice[1..])
	}
}
\end{lstlisting}


\section{Функциональные тесты}

В данном разделе представлены функциональные тесты для разработанных алгоритмов: нерекурсивный алгоритм подсчёта единиц в последовательности (таблица 3.1), рекурсивный алгоритм подсчёта единиц в последовательности (таблица 3.2).

\begin{table}[h!]
	\centering
	\caption{Функциональные тесты для нерекурсивного алгоритма}
	\label{table:functional_tests_iterative}
	\begin{tabular}{|p{1.5cm}|p{3.5cm}|p{5cm}|p{5cm}|}
		\hline
		\textbf{№} & \textbf{Входная последовательность} & \textbf{Ожидаемый результат} & \textbf{Полученный результат} \\
		\hline
		1 & 2 & 0 & 0 \\
		\hline
		2 & 1 2 & 1 & 1 \\
		\hline
		3 & 0 2 & 0 & 0 \\
		\hline
		4 & 1 1 1 2 & 3 & 3 \\
		\hline
		5 & 0 0 0 2 & 0 & 0 \\
		\hline
		6 & 1 0 1 0 2 & 2 & 2 \\
		\hline
		7 & 0 1 0 1 0 2 & 2 & 2 \\
		\hline
		8 & 1 1 0 2 1 1 & 2 & 2 \\
		\hline
		9 & 1 0 1 0 1 0 1 2 & 4 & 4 \\
		\hline
		10 & a & Некорректный ввод: 'a' & Некорректный ввод: 'a' \\
		\hline
		11 & 1 3 & Недопустимое число 3. Разрешены только 0, 1, 2
		 & Недопустимое число 3. Разрешены только 0, 1, 2
		 \\
		\hline
	\end{tabular}
\end{table}

\begin{table}[h!]
	\centering
	\caption{Функциональные тесты для рекурсивного алгоритма}
	\label{table:functional_tests_recursive}
	\begin{tabular}{|p{1.5cm}|p{3.5cm}|p{5cm}|p{5cm}|}
		\hline
		\textbf{№} & \textbf{Входная последовательность} & \textbf{Ожидаемый результат} & \textbf{Полученный результат} \\
		\hline
		1 & 2 & 0 & 0 \\
		\hline
		2 & 1 2 & 1 & 1 \\
		\hline
		3 & 0 2 & 0 & 0 \\
		\hline
		4 & 1 1 1 2 & 3 & 3 \\
		\hline
		5 & 0 0 0 2 & 0 & 0 \\
		\hline
		6 & 1 0 1 0 2 & 2 & 2 \\
		\hline
		7 & 0 1 0 1 0 2 & 2 & 2 \\
		\hline
		8 & 1 1 0 2 1 1 & 2 & 2 \\
		\hline
		9 & 1 0 1 0 1 0 1 2 & 4 & 4 \\
		\hline
		10 & a & Некорректный ввод: 'a' & Некорректный ввод: 'a' \\
		\hline
		11 & 1 3 & Недопустимое число 3. Разрешены только 0, 1, 2
		& Недопустимое число 3. Разрешены только 0, 1, 2
		\\
		\hline
	\end{tabular}
\end{table}

Все функциональные тесты были успешно пройдены.

\section*{Вывод}

В данном разделе были реализованы алгоритмы нерекурсивный алгоритм подсчёта единиц в последовательности и рекурсивный алгоритм подсчёта единиц в последовательности. Все алгоритмы включают проверку корректности входных данных и обработку ошибок.

Проведённые функциональные тесты подтвердили корректность работы всех реализованных алгоритмов.

\clearpage
