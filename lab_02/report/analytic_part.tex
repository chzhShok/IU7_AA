\chapter{Аналитическая часть}

\section{Математическая модель}

Пусть дана последовательность $S = \{s_1, s_2, ..., s_n, 2\}$, где $s_i \in \{0, 1\}$. Требуется найти значение:
$$count = \sum_{i=1}^{n} \delta(s_i, 1)$$
где $\delta(x, y)$ – функция Кронекера, равная 1 при $x = y$ и 0 в противном случае.

\section{Рекурсия}

\subsection{Определение рекурсии}

\textbf{Рекурсия} – это метод определения функций или решения задач, при котором функция вызывает саму себя непосредственно или через другие функции~\cite{goloveshin}. 

\vspace{10pt}

Рекурсивная функция $f$ определяется через~\cite{goloveshin,barron}:
\begin{itemize}
	\item \textbf{базовый случай} (терминальное условие): $f(x_0) = c$, где $x_0$ – терминальное значение;
	\item \textbf{рекурсивный случай}: $f(x) = g(x, f(h(x)))$, где $h(x)$ – функция, уменьшающая задачу.
\end{itemize}

\vspace{10pt}

Корректность рекурсивного алгоритма обеспечивается выполнением условий~\cite{cormen}:
\begin{itemize}
	\item \textbf{существование базового случая}: функция завершается при обнаружении терминатора 2 или пустой последовательности;
	\item \textbf{сходимость}: на каждом шаге размер задачи уменьшается ($|rest(S)| < |S|$);
	\item \textbf{правильность}: результат вычисляется как сумма вклада текущего элемента и результата для оставшейся последовательности.
\end{itemize}

\subsection{Математическая модель рекурсивного алгоритма подсчёта}

Для задачи подсчёта единиц в последовательности рекурсивная функция $count(S)$ определяется как:

\[
count(S) = 
\begin{cases}
	0, & \text{если } S = \emptyset \text{ или } first(S) = 2 \\
	1 + count(rest(S)), & \text{если } first(S) = 1 \\
	0 + count(rest(S)), & \text{если } first(S) = 0
\end{cases}
\]

где:
\begin{itemize}
	\item $first(S)$ – первый элемент последовательности $S$;
	\item $rest(S)$ – последовательность без первого элемента;
	\item $\emptyset$ – пустая последовательность.
\end{itemize}


\section*{Вывод}

В аналитической части:
\begin{itemize} 
	\item проведён анализ математической модели задачи, формализованной через сумму функций Кронекера;
	\item представлено строгое математическое описание рекурсивного метода с выделением базового и рекурсивного случаев;
	\item определены условия корректности рекурсивного алгоритма, обеспечивающие его сходимость и правильность работы.
\end{itemize}

\clearpage