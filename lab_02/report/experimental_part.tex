\chapter{Исследовательская часть}

\section{Характеристики ЭВМ}
Замеры проводились на устройстве со следующими характеристиками:

\begin{itemize} 
	\item процессор: Apple M4 Pro;
	\item оперативная память: 24 Гб;
	\item операционная система: macOS Sequoia 15.6.1.
\end{itemize}

Замеры времени проводились, когда ноутбук был загружен только системными приложениями.

\section{Результаты замера времени}

В данном разделе представлены результаты измерения времени выполнения алгоритмов подсчёта единиц в последовательности. 

Замеры проводились для последовательностей размером от 100 до 2000 символов. Каждый замер выполнялся 1000 раз.

\begin{table}[h!]
	\centering
	\begin{tabular}{|c|c|c|}
		\hline
		\textbf{Размер} & \textbf{Рекурсия (мкс)} & \textbf{Итерация (мкс)} \\
		\hline
		100 & 2.649 & 0.889 \\
		\hline
		200 & 5.395 & 1.453 \\
		\hline
		300 & 6.700 & 1.695 \\
		\hline
		400 & 7.555 & 2.109 \\
		\hline
		500 & 8.273 & 2.068 \\
		\hline
		600 & 8.626 & 2.362 \\
		\hline
		700 & 9.629 & 2.488 \\
		\hline
		800 & 10.247 & 2.783 \\
		\hline
		900 & 10.489 & 2.816 \\
		\hline
		1000 & 11.585 & 2.957 \\
		\hline
		1100 & 12.170 & 3.235 \\
		\hline
		1200 & 12.382 & 3.464 \\
		\hline
		1300 & 14.295 & 3.764 \\
		\hline
		1400 & 14.875 & 4.018 \\
		\hline
		1500 & 16.320 & 4.551 \\
		\hline
		1600 & 17.899 & 4.787 \\
		\hline
		1700 & 18.396 & 5.036 \\
		\hline
		1800 & 20.388 & 5.435 \\
		\hline
		1900 & 20.858 & 5.766 \\
		\hline
		2000 & 22.215 & 6.014 \\
		\hline
	\end{tabular}
	\caption{Результаты измерения времени}
	\label{table:time_measurements}
\end{table}

\begin{figure}[h!]\label{timing} 
	\center{\includegraphics[scale=0.5]{./images/algorithm_comparison.png}} 
	\caption{График замера времени}
\end{figure}

\section*{Вывод}

Согласно результатм исследования, нерекурсивный алгоритм работает быстрее рекурсивного алгоритма. Эта разница становится заметнее с увеличением размера последовательности.

\clearpage
