\section*{ВВЕДЕНИЕ}

Целью данной работы является исследование рекурсивного и нерекурсивного алгоритмов подсчета количества единиц в последовательности из нулей и единиц, завершающейся числом 2. 

Для достижения поставленной цели необходимо решить следующие задачи:

\begin{itemize}
	\item разработать рекурсивный и нерекурсивный алгоритмы решения задачи;
	\item описать средства разработки и инструменты замера процессорного времени выполнения реализации алгоритмов;
	\item реализовать разработанные алгоритмы;
	\item выполнить тестирование реализации алгоритмов на корректность работы;
	\item выполнить теоретическую оценку затрачиваемой реализацией каждого алгоритма памяти, включая анализ высоты дерева рекурсивных вызовов;
	\item выполнить замеры процессорного времени выполнения реализации алгоритмов в зависимости от варьируемого размера входных данных;
	\item оценить трудоёмкость двух алгоритмов в худшем случае;
	\item сравнить результаты замеров процессорного времени и оценки трудоемкости;
	\item провести сравнительный анализ реализации рекурсивного и нерекурсивного алгоритмов по критериям ёмкостной и временной эффективности.
\end{itemize}

\clearpage