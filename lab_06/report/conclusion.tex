\ssr{ЗАКЛЮЧЕНИЕ}

В ходе выполнения лабораторной работы поставленная цель была достигнута, а также
были решены следующие задачи:
\begin{itemize}
	\item сформулирована задача коммивояжёра для полносвязного неориентированного графа с требованием нахождения гамильтонова пути (незамкнутого маршрута);
	\item разработан и реализован алгоритм полного перебора для точного решения задачи;
	\item разработан и реализован муравьиный алгоритм без элитных муравьёв;
	\item проведена параметризация муравьиного алгоритма по трём основным параметрам ($\alpha$, $\beta$, $\rho$) и числу итераций;
	\item выполнена оценка трудоёмкости разработанных алгоритмов и сравнительный анализ их временных характеристик;
	\item сформулированы рекомендации по настройке муравьиного алгоритма и выбору метода решения в зависимости от размера задачи.
\end{itemize}

\vspace{10pt}

По результатам исследования установлено:
\begin{itemize}
	\item алгоритм полного перебора обладает факториальной трудоёмкостью $O(n! \cdot n)$ и обеспечивает точное решение задачи, но практически применим только для графов малой размерности;
	\item муравьиный алгоритм имеет полиномиальную сложность порядка $O(t_{max} \cdot n^3)$ и позволяет находить близкие к оптимальным решения для графов существенно большей размерности;
	\item результаты параметризации показали, что наиболее устойчивую работу муравьиного алгоритма обеспечивают параметры $\alpha \in [0{,}1; 0{,}5]$, $\beta = 1 - \alpha$, $\rho \in [0{,}1; 0{,}5]$ и $t_{max} \in [300; 500]$.
\end{itemize}
