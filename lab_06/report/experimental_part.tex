\chapter{Исследовательская часть}

\section{Характеристики ЭВМ}
Замеры проводились на устройстве со следующими характеристиками:

\begin{itemize} 
	\item процессор: Apple M4 Pro;
	\item количество логических ядер: 12;
	\item количество ядер: 12;
	\item оперативная память: 24 Гб;
	\item операционная система: macOS Sequoia 15.6.1.
\end{itemize}

Замеры времени проводились, когда ноутбук был загружен только системными приложениями.

\section{Сравнение времени работы}

Для оценки временной эффективности алгоритмов полного перебора и муравьиного алгоритма были проведены замеры на случайно сгенерированных полносвязных неориентированных графах с целочисленными весами рёбер. Для каждого размера графа выполнялось по нескольку запусков и усреднялось процессорное время.

Результаты приведены в таблице~\ref{tbl:time_cmp}.

\begin{table}[h]
	\begin{center}
		\caption{\label{tbl:time_cmp}Сравнение алгоритмов по времени выполнения}
		\begin{tabular}{|l|l|l|}
			\hline
			Размер &
			Время полного перебора, с &
			Время муравьиного алгоритма, с \\
			\hline
			     2 &                0{,}000010 &                    0{,}000316 \\
			\hline
			     3 &                0{,}000003 &                    0{,}001013 \\
			\hline
			     4 &                0{,}000010 &                    0{,}002391 \\
			\hline
			     5 &                0{,}000043 &                    0{,}004671 \\
			\hline
			     6 &                0{,}000284 &                    0{,}008640 \\
			\hline
			     7 &                0{,}002464 &                    0{,}013845 \\
			\hline
			     8 &                0{,}020792 &                    0{,}021348 \\
			\hline
			     9 &                0{,}209074 &                    0{,}031641 \\
			\hline
			    10 &                2{,}313401 &                    0{,}045135 \\
			\hline
		\end{tabular}
	\end{center}
\end{table}

На основе полученных данных был построен график зависимости времени работы алгоритмов от количества вершин (рисунок~\ref{fig:time_graph}).

\begin{figure}[h]
	\centering
	\includegraphics[scale=0.7]{images/time_graph.png}
	\caption{Сравнение алгоритмов по времени выполнения}
	\label{fig:time_graph}
\end{figure}

\clearpage

\section{Параметризация муравьиного алгоритма}

Была выполнена параметризация муравьиного алгоритма по трём основным параметрам (коэффициенты влияния феромона и эвристики, коэффициент испарения феромона), а также по числу итераций. Замеры проводились на фиксированном полносвязном неориентированном графе малого размера; для каждой комбинации параметров вычислялось отклонение длины маршрута, найденного муравьиным алгоритмом, от оптимального значения, полученного методом полного перебора. Результаты параметризации приведены в таблице~\ref{tbl:parametrization} Приложения~А.

\section{Вывод}

При увеличении числа вершин трудоёмкость алгоритма полного перебора быстро возрастает и делает практическое применение невозможным. Муравьиный алгоритм на тех же тестовых данных показывает значительно более плавный рост времени выполнения и, начиная примерно с $n \approx 8$ вершин, становится сопоставим по скорости или быстрее полного перебора, а при дальнейшем увеличении размера графа его преимущество по времени только усиливается.

По полученной таблице параметризации наилучшие результаты дают наборы с $\alpha \in [0{,}1; 0{,}5]$, $\beta = 1 - \alpha$, $\rho \in [0{,}1; 0{,}5]$ и $t_{max} \in [300; 500]$, при которых во многих случаях удаётся получить маршрут с нулевым отклонением.

\clearpage
