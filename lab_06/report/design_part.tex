\chapter{Конструкторская часть}

\section{Требования к реализации}

Разрабатываемое программное обеспечение представляет собой консольное приложение на языке Python, работающее в нескольких режимах:
\begin{itemize}
	\item \textbf{режим одиночного запуска} --- пользователь выбирает источник данных, задаёт параметры муравьиного алгоритма, после чего программа выводит найденный маршрут и его длину для алгоритма полного перебора и/или муравьиного алгоритма;
	\item \textbf{режим замера времени} --- по заданному диапазону размеров графа выполняется серия запусков обоих алгоритмов, измеряется время работы и формируется таблица для последующего анализа и построения графиков;
	\item \textbf{режим параметризации} --- проводится исследование влияния параметров муравьиного алгоритма.
\end{itemize}

\section{Описание алгоритмов}

На рисунке~\ref{fig:design_bruteforce} представлена схема алгоритма полного перебора, на рисунках~\ref{fig:design_ant_1}--\ref{fig:design_ant_2} --- схема муравьиного алгоритма.

\begin{figure}[h]
	\centering
	\includegraphics[scale=0.5]{images/brute_force.png}
	\caption{Описание алгоритма полного перебора}
	\label{fig:design_bruteforce}
\end{figure}

\clearpage

\begin{figure}[h]
	\centering
	 \includegraphics[scale=0.8]{images/ant_alg_1}
	\caption{Муравьиный алгоритм, часть 1}
	\label{fig:design_ant_1}
\end{figure}

\clearpage

\begin{figure}[h]
	\centering
	\includegraphics[scale=0.7]{images/ant_alg_2}
	\caption{Муравьиный алгоритм, часть 2}
	\label{fig:design_ant_2}
\end{figure}

\clearpage

\section*{Вывод}

В данном разделе были представлены схемы для алгоритма полного перебора и муравьиного алгоритма поиска кратчайших путей в графе.
\clearpage


