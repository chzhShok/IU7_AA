\chapter{Аналитическая часть}

\section{Постановка задачи коммивояжёра}

Пусть задан неориентированный взвешенный граф $G = (V, E)$, где $V = \{v_1, v_2, \ldots, v_n\}$~--- множество вершин, соответствующих городам, а $E \subseteq V \times V$~--- множество рёбер, соответствующих возможным путям между городами. Каждому ребру $(i, j) \in E$ сопоставлен вес $w_{ij} \in \mathbb{R}^+$, определяющий время перемещения между вершинами $i$ и $j$. Для неориентированного графа матрица весов $W = \|w_{ij}\|_{n \times n}$ является симметричной, то есть $w_{ij} = w_{ji}$.

Классическая задача коммивояжёра формулируется как поиск гамильтонова цикла минимальной длины, то есть замкнутого маршрута, проходящего через каждую вершину графа ровно один раз и возвращающегося в исходную вершину~\cite{lit1}.

В настоящей работе рассматривается модификация задачи: требуется найти \textbf{гамильтонов путь} минимальной длины, то есть \textbf{незамкнутый маршрут}, проходящий через каждую вершину графа ровно один раз без возврата в начальный город.

Пусть $\pi = (\pi_1, \pi_2, \ldots, \pi_n)$~--- перестановка номеров вершин графа, определяющая порядок их посещения. Тогда длина (время) незамкнутого маршрута определяется как
\begin{equation}
	L(\pi) = \sum_{i=1}^{n-1} w_{\pi_i \pi_{i+1}}.
\end{equation}

Необходимо найти такую перестановку $\pi^\ast$, что
\begin{equation}
	L(\pi^\ast) = \min_{\pi} L(\pi).
\end{equation}

\section{Метод полного перебора}

Алгоритм полного перебора (brute force) представляет собой точный метод решения задачи коммивояжёра, основанный на анализе всех возможных маршрутов~\cite{lit2}.

Основные этапы алгоритма:
\begin{enumerate}
	\item сгенерировать все возможные перестановки вершин графа (размер пространства поиска составляет $(n-1)!$ маршрутов);
	\item для каждой перестановки $\pi$ вычислить длину незамкнутого маршрута $L(\pi)$;
	\item выбрать перестановку с минимальным значением функции $L(\pi)$.
\end{enumerate}

В отличие от классической постановки с циклом, в данной работе длина маршрута вычисляется только по переходам между последовательными вершинами пути и не включает возврат из последней вершины в первую.

Алгоритм полного перебора гарантированно находит оптимальное решение, однако имеет факториальную трудоёмкость $O(n! \cdot n)$ и потому пригоден только для графов небольшой размерности.

\section{Муравьиный алгоритм}

Муравьиный алгоритм относится к классу стохастических оптимизационных методов, инспирированных поведением муравьиных колоний при поиске кратчайших путей между гнездом и источниками пищи~\cite{lit3}. Муравьи взаимодействуют опосредованно, через феромонные следы, усиливая успешные маршруты и постепенно <<забывая>> неудачные.

В данной работе используется модификация муравьиного алгоритма для поиска гамильтонова пути без использования элитных муравьёв.

\subsection{Феромонная модель и эвристика}

Каждому ребру $(i,j)$ сопоставлен уровень феромона $\tau_{ij}$, отражающий накопленный <<опыт>> успешного использования этого ребра муравьями.

Эвристическая информация задаётся как \emph{видимость} ребра
\begin{equation}
	\eta_{ij} = \frac{1}{w_{ij}},
\end{equation}
которая тем больше, чем короче ребро.

\subsection{Правило выбора следующей вершины}

При построении маршрута каждый муравей находится в некоторой текущей вершине $i$ и имеет множество допустимых для перехода вершин $J_k(i)$, не содержащих уже посещённые вершины. Вероятность перехода муравья $k$ из вершины $i$ в вершину $j$ определяется соотношением~\cite{lit3}:
\begin{equation}
	P_{ij}^k =
	\frac{\left[\tau_{ij}\right]^\alpha \cdot \left[\eta_{ij}\right]^\beta}
	{\sum\limits_{l \in J_k(i)} \left[\tau_{il}\right]^\alpha \cdot \left[\eta_{il}\right]^\beta},
\end{equation}
где:
\begin{itemize}
	\item $\alpha \geqslant 0$~--- параметр влияния феромона;
	\item $\beta \geqslant 0$~--- параметр влияния эвристики;
	\item $J_k(i)$~--- множество непосещённых вершин, допустимых для перехода.
\end{itemize}

\subsection{Правило обновления феромонов}

После того как все муравьи построили свои маршруты, выполняется обновление уровней феромонов на рёбрах. Пусть $L_k$~--- длина маршрута $k$-го муравья, а $m$~--- число муравьёв. Тогда суммарное количество феромона, добавляемое на ребро $(i,j)$, вычисляется как
\begin{equation}
	\Delta \tau_{ij} = \sum_{k=1}^{m}
	\begin{cases}
		\displaystyle \frac{Q}{L_k}, & \text{если ребро $(i,j)$ входит в маршрут муравья $k$,}\\[1ex]
		0, & \text{иначе,}
	\end{cases}
\end{equation}
где $Q$~--- параметр, пропорциональный среднему весу рёбер графа.

Обновление феромона задаётся правилом
\begin{equation}
	\tau_{ij}(t+1) = (1 - \rho)\,\tau_{ij}(t) + \Delta \tau_{ij},
\end{equation}
где $\rho \in (0, 1]$~--- коэффициент испарения феромона.

\section{Вывод}

В данном разделе была сформулирована задача коммивояжёра в постановке незамкнутого маршрута на неориентированном графе и рассмотрены теоретические основы двух подходов к её решению. Алгоритм полного перебора обеспечивает точное нахождение минимального по длине гамильтонова пути, но обладает факториальной трудоёмкостью и применим только для графов малой размерности. Муравьиный алгоритм использует феромонную модель и эвристическую видимость рёбер для вероятностного построения маршрутов и позволяет получать приближённые решения при существенно меньшей трудоёмкости.

\clearpage
