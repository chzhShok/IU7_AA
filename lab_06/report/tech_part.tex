\chapter{Технологическая часть}

\section{Средства реализации}

Для реализации алгоритмов был выбран язык Python~3, так как он соответствует требованиям лабораторной работы с использованием следующих библиотек:

Замеры времени выполнения проводились с использованием функции \texttt{process\_time} модуля \texttt{time}, измеряющей процессорное время текущего процесса. Разработка проводилась в интегрированной среде разработки \texttt{PyCharm}.

\section{Реализация алгоритмов}

В листинге~\ref{lst:bruteforce} приведён фрагмент кода алгоритма полного перебора, а реализация муравьиного алгоритма вынесена в приложение (листинг~\ref{lst:ant}).

\begin{lstlisting}[label=lst:bruteforce,caption={Алгоритм полного перебора},captionpos=b]
def brute_force_path(matrix: np.ndarray) -> tuple[float, list[int]]:
    n = matrix.shape[0]
    indices = list(range(n))

    best_len = float("inf")
    best_path: list[int] = []

    for perm in it.permutations(indices):
        length = 0.0
        for i in range(n - 1):
            length += float(matrix[perm[i], perm[i + 1]])

        if length < best_len:
            best_len = length
            best_path = list(perm)

    return best_len, best_path
\end{lstlisting}

\section{Подготовка входных данных}

Для построения матрицы расстояний между странами Африки задаётся фиксированный набор из 11 стран, формируется квадратная матрица $11 \times 11$ с нулями на диагонали. Для каждой пары стран по их координатам (широта, долгота) считается расстояние и умножается на коэффициент местности: для двух пустынных государств расстояние увеличивается в $1{,}5$ раза, для маршрутов с участием Мадагаскара уменьшается до $0{,}7$ от сухопутного аналога. В результате получаются значения от $950$ до $5571$: например, <<Алжир~--- Ливия>> даёт $2326$, <<Алжир~--- Египет>>~--- $4309$, <<Мали~--- Египет>>~--- $5571$, <<Нигер~--- Нигерия>>~--- $950$, <<Нигер~--- Чад>>~--- $1740$. Для связей с Мадагаскаром значения заметно меньше: <<Кения~--- Мадагаскар>> даёт $1612$, <<Эфиопия~--- Мадагаскар>>~--- $2227$. Округлённая до целых матрица сохраняется в текстовый файл и далее используется как входные данные для функциональных тестов и параметризации.

\section{Функциональные тесты}

Для проверки корректности работы программной реализации были проведены функциональные тесты на матрицах расстояний между странами Африки, построенных по географическим координатам и скорректированных с учётом типа местности.

В таблице~\ref{tbl:func_tests} приведены примеры тестов.

\begin{table}[h]
	\centering
	\caption{\label{tbl:func_tests}Функциональные тесты на матрицах карты Африки}
	\small
	\begin{tabular}{|c|c|c|}
		\hline
		\textbf{Матрица расстояний} & \textbf{Страны} & \textbf{Результат} \\
		\hline
		\begin{tabular}{@{}c@{}}
			$\begin{pmatrix}
				0    & 2326 & 4309 & 1949 & 1999 & 3369 \\
				2326 & 0    & 2025 & 3587 & 2028 & 1830 \\
				4309 & 2025 & 0    & 5571 & 3822 & 2665 \\
				1949 & 3587 & 5571 & 0    & 1920 & 3649 \\
				1999 & 2028 & 3822 & 1920 & 0    & 1740 \\
				3369 & 1830 & 2665 & 3649 & 1740 & 0
			\end{pmatrix}$
		\end{tabular}
		&
		\begin{tabular}{@{}c@{}}
			1~--- Алжир\\
			2~--- Ливия\\
			3~--- Египет\\
			4~--- Мали\\
			5~--- Нигер\\
			6~--- Чад
		\end{tabular}
		&
		\begin{tabular}{@{}c@{}}
			Оптимальный незамкнутый маршрут:\\
			$1 \to 4 \to 5 \to 6 \to 2 \to 3$\\
			Минимальная длина: $9464$
		\end{tabular} \\
		\hline
		\begin{tabular}{@{}c@{}}
			$\begin{pmatrix}
				0    & 1740 & 3646 & 950  & 3806 & 3624 \\
				1740 & 0    & 1907 & 1302 & 2720 & 2464 \\
				3646 & 1907 & 0    & 2388 & 1665 & 1195 \\
				950  & 1302 & 2388 & 0    & 3391 & 3492 \\
				3806 & 2720 & 1665 & 3391 & 0    & 1059 \\
				3624 & 2464 & 1195 & 3492 & 1059 & 0
			\end{pmatrix}$
		\end{tabular}
		&
		\begin{tabular}{@{}c@{}}
			1~--- Нигер\\
			2~--- Чад\\
			3~--- Судан\\
			4~--- Нигерия\\
			5~--- Кения\\
			6~--- Эфиопия
		\end{tabular}
		&
		\begin{tabular}{@{}c@{}}
			Оптимальный незамкнутый маршрут:\\
			$1 \to 4 \to 2 \to 3 \to 6 \to 5$\\
			Минимальная длина: $6413$
		\end{tabular} \\
		\hline
		\begin{tabular}{@{}c@{}}
			Матрица 11x11 (см. Приложение~\ref{app:big_matrix})
		\end{tabular}
		&
		\begin{tabular}{@{}c@{}}
			1~--- Алжир\\
			2~--- Ливия\\
			3~--- Египет\\
			4~--- Мали\\
			5~--- Нигер\\
			6~--- Чад\\
			7~--- Судан\\
			8~--- Нигерия\\
			9~--- Кения\\
			10~--- Эфиопия\\
			11~--- Мадагаскар
		\end{tabular}
		&
		\begin{tabular}{@{}c@{}}
			Оптимальный незамкнутый маршрут:\\
			$1 \to 4 \to 5 \to 8 \to 6 \to 2 \to 3 \to$ \\
			$7 \to 10 \to 9 \to 11$\\
			Минимальная длина: $16172$
		\end{tabular} \\
		\hline
	\end{tabular}
\end{table}

Во всех трёх тестах результаты муравьиного алгоритма при подобранных параметрах совпали или оказались очень близки к результатам полного перебора, что подтверждает корректность реализации.

\clearpage

\section{Оценка трудоёмкости алгоритмов}

\subsection{Алгоритм полного перебора}

Трудоёмкость алгоритма полного перебора рассчитывается следующим образом:

\begin{enumerate}
	\item инициализация переменных: $f_{init} = 3$;
	\item генерация всех перестановок ($(n-1)!$ итераций):
	\[
	f_{permutations} = (n-1)! \cdot f_{iteration}
	\]
	\item обработка одной перестановки:
	\[
	f_{iteration} = f_{length} + f_{compare} + f_{update}
	\]
	\item вычисление длины пути: $f_{length} = n-1$;
	\item сравнение с минимальной длиной: $f_{compare} = 1$;
	\item обновление лучшего пути: $f_{update} = n$.
\end{enumerate}

Для каждой перестановки:
\[
f_{iteration} = (n-1) + 1 + n = 2n
\]

Общая трудоёмкость:
\[
f_{brute\_force} = 3 + (n-1)! \cdot (2n) \approx O(n! \cdot n)
\]

\subsection{Муравьиный алгоритм}

Трудоёмкость муравьиного алгоритма рассчитывается следующим образом:

\begin{enumerate}
	\item инициализация матрицы феромонов: $f_{init\_pheromone} = n^2$;
	\item инициализация матрицы видимости: $f_{init\_visibility} = n^2$;
	\item основной цикл ($t_{max}$ итераций):
	\[
	f_{main\_loop} = t_{max} \cdot (f_{ants} + f_{update})
	\]
	\item цикл по муравьям ($n$ муравьёв):
	\[
	f_{ants} = n \cdot f_{ant\_path}
	\]
	\item построение пути одного муравья ($n-1$ шагов):
	\[
	f_{ant\_path} = (n-1) \cdot (f_{probabilities} + f_{next}) + f_{calc\_length}
	\]
	\item вычисление вероятностей перехода: $f_{probabilities} = 2n$;
	\item выбор следующей вершины: $f_{next} = n$;
	\item расчёт длины пути: $f_{calc\_length} = n$;
	\item обновление феромонов: $f_{update} = n^2 \cdot n = n^3$.
\end{enumerate}

Для одного муравья:
\[
f_{ant\_path} = (n-1) \cdot (2n + n) + n = (n-1) \cdot 3n + n = 3n^2 - 2n
\]

Для всех муравьёв на одной итерации:
\[
f_{ants} = n \cdot (3n^2 - 2n) = 3n^3 - 2n^2
\]

Общая трудоёмкость одной итерации:
\[
f_{day} = (3n^3 - 2n^2) + n^3 = 4n^3 - 2n^2
\]

Общая трудоёмкость алгоритма:
\[
f_{ant} = 2n^2 + t_{max} \cdot (4n^3 - 2n^2) \approx O(t_{max} \cdot n^3)
\]

\section*{Вывод}

В технологической части были выбраны средства реализации, приведены фрагменты кода реализованных алгоритмов, проведены функциональные тесты и выполнена оценка трудоёмкости методов полного перебора и муравьиного алгоритма. Алгоритм полного перебора обладает факториальной сложностью и подходит лишь для малых графов, тогда как муравьиный алгоритм имеет полиномиальную сложность $O(t_{\max} \cdot n^3)$ и может применяться к задачам существенно большей размерности.

\clearpage
