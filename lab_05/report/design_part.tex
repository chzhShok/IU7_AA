\chapter{Конструкторская часть}

\section{Требования к реализации}

К программе предъявлены следующие функциональные требования.

\vspace{10pt}

\textbf{Входные данные}
\begin{itemize}[label=---]
	\item граф в формате DOT, содержащий ориентированный взвешенный граф;
	\item стартовая вершина -- имя вершины, от которой ищутся пути;
	\item целевые вершины -- список имён вершин, до которых ищутся пути;
	\item количество заявок $N$ -- целое число, определяющее, сколько раз алгоритм будет запущен для заданного графа.
\end{itemize}

\vspace{10pt}

\textbf{Выходные данные}
\begin{itemize}[label=---]
	\item имя стартовой вершины;
	\item массив имён целевых вершин;
	\item время суммарного выполнения алгоритма для $N$ запусков (выводится в микросекундах в стандартный поток вывода);
	\item объект с расстояниями до каждой целевой вершины;
	\item объект с информацией о кратчайшем пути, содержащий: имя целевой вершины с минимальным расстоянием, длину кратчайшего пути, массив имён вершин, составляющих путь от стартовой до целевой вершины;
	\item лог о работе программы.
\end{itemize}

\vspace{10pt}

\textbf{Функциональные требования}
\begin{itemize}[label=---]
	\item поддержка загрузки графов из файлов формата DOT;
	\item реализация последовательного алгоритма Дейкстры;
	\item реализация параллельного алгоритма Дейкстры с использованием нативных потоков;
	\item реализация конвейерной обработки набора заявок на поиск кратчайших путей с использованием трёх обслуживающих устройств и блокирующих очередей между ними;
	\item обработка некорректных входных данных с выводом сообщений об ошибках;
	\item вывод результатов в человекочитаемом текстовом формате (отчёт по каждой заявке в отдельный файл) для последующей обработки и анализа;
	\item замер времени выполнения алгоритмов.
\end{itemize}

\vspace{10pt}

\textbf{Режимы работы} 
\begin{itemize}[label=---]
	\item линейный режим -- выполнение последовательного алгоритма Дейкстры;
	\item конвейерный режим -- последовательная обработка набора заявок на трёх обслуживающих устройствах, где на второй ступени выполняется параллельный алгоритм Дейкстры.
\end{itemize}

\section {Разработка алгоритмов}

Раздел содержит схемы алгоритмов, описывающие следующие алгоритмы:
алгоритм линейной обработки~\ref{fig:linear}, алгоритм конвейерной обработки~\ref{fig:pipeline}, алгоритм обслуживающего устройства 1~\ref{fig:ou1}, алгоритм обслуживающего устройства 2~\ref{fig:ou2}, алгоритм обслуживающего устройства 3~\ref{fig:ou3}.

\begin{figure}[H]
	\center{\includegraphics[scale=0.37]{./images/linear.png}} 
	\caption{Схема линейной обработки}
	\label{fig:linear} 
\end{figure}

\begin{figure}[H]
	\center{\includegraphics[scale=0.5]{./images/pipeline.png}} 
	\caption{Схема конвейерной обработки}
	\label{fig:pipeline} 
\end{figure}

\begin{figure}[H]
	\center{\includegraphics[scale=0.5]{./images/ou1.png}} 
	\caption{Схема обслуживающего устройства 1}
	\label{fig:ou1} 
\end{figure}

\begin{figure}[H]
	\center{\includegraphics[scale=0.5]{./images/ou2.png}} 
	\caption{Схема обслуживающего устройства 2}
	\label{fig:ou2} 
\end{figure}

\begin{figure}[H]
	\center{\includegraphics[scale=0.5]{./images/ou3.png}} 
	\caption{Схема обслуживающего устройства 3}
	\label{fig:ou3} 
\end{figure}

\section*{Вывод}

В данном разделе были разработаны алгоритмы линейной, конвейерной обработки и алгоритмы обслуживающего устройства 1, обслуживающего устройства 2, обслуживающего устройства 3. Для каждого из них представлены схемы алгоритмов, описывающие логику работы.

\clearpage
