\chapter{Исследовательская часть}

\section{Характеристики ЭВМ}
Замеры проводились на устройстве со следующими характеристиками:

\begin{itemize} 
	\item процессор: Apple M4 Pro;
	\item количество логических ядер: 12;
	\item количество физических ядер: 12;
	\item оперативная память: 24 Гб;
	\item операционная система: macOS Sequoia 15.6.1.
\end{itemize}

Замеры времени проводились, когда ноутбук был загружен только системными приложениями.

\section{Замер времени}

В данном разделе представлены результаты измерения среднего времени выполнения конвейерного алгоритма обработки заявок.

Замеры проводились для различного количества заявок \(N \in \{25, 50, 75, 100, 125\}\) для графов размером от 500, 800, 1000, 1500, 2000 вершин и двух конфигураций исполнения:
\begin{itemize}
	\item линейный вариант;
	\item конвейерный вариант.
\end{itemize}

Для каждой конфигурации замер повторялся 5 раз, в таблице приведено среднее время выполнения в миллисекундах. Результаты приведены в таблице~\ref{table:time_measurements}, графики по таблице представлен на рисунке~\ref{fig:experiments_time_avg}.

\begin{table}[H]
	\centering
	\small
	\caption{Результаты измерения среднего времени выполнения (мс)}
	\begin{tabular}{|c|r|r|}
		\hline
		\textbf{N} & \textbf{Линейный, мс} & \textbf{Конвейерный, мс} \\
		\hline
		25 & 41122.748 & 39066.611 \\
		\hline
		50 & 82146.180 & 77217.409 \\
		\hline
		75 & 123156.066 & 114535.142 \\
		\hline
		100 & 164241.412 & 151102.499 \\
		\hline
		125 & 205278.415 & 186803.357 \\
		\hline
	\end{tabular}
	\label{table:time_measurements}
\end{table}

\begin{figure}[H]
	\center{\includegraphics[scale=0.7]{./images/experiments_time_avg.png}} 
	\caption{Сравнение времени выполнения реализации линейного и конвейерного алгоритмов}
	\label{fig:experiments_time_avg} 
\end{figure}

\section{Логирование событий}

В листинге~\ref{lst:log} представлен пример логирования событий для графа на 3000 вершин при количестве заявок \(N = 4\). Из логов следует, что события на разных обслуживающих устройствах происходят параллельно. Например, в интервале \([4244982, 4246195]\) мкс одновременно работают ОУ1 (заявка~1) и ОУ2 (заявка~0), а в дальнейшем аналогичное перекрытие наблюдается и между другими заявками.

Анализ временных меток также показывает конвейерный характер обработки: каждая заявка последовательно проходит через ОУ1, ОУ2 и ОУ3, при этом, когда одна заявка уже обрабатывается на ОУ3, другая в это время может находиться на ОУ2, а третья -- на ОУ1.

\begin{lstlisting}[language=C++,caption={Логирование событий},label={lst:log}]
[15] START | Заявка#0 | ОУ1
[4244972] END | Заявка#0 | ОУ1
[4244975] START | Заявка#1 | ОУ1
[4244982] START | Заявка#0 | ОУ2
[4246195] END | Заявка#0 | ОУ2
[4246200] START | Заявка#0 | ОУ3
[4247599] END | Заявка#0 | ОУ3
[8411120] END | Заявка#1 | ОУ1
[8411122] START | Заявка#2 | ОУ1
[8411129] START | Заявка#1 | ОУ2
[8412229] END | Заявка#1 | ОУ2
[8412233] START | Заявка#1 | ОУ3
[8413450] END | Заявка#1 | ОУ3
[12593969] END | Заявка#2 | ОУ1
[12593972] START | Заявка#3 | ОУ1
[12593978] START | Заявка#2 | ОУ2
[12595146] END | Заявка#2 | ОУ2
[12595150] START | Заявка#2 | ОУ3
[12595324] END | Заявка#2 | ОУ3
[16874896] END | Заявка#3 | ОУ1
[16874905] START | Заявка#3 | ОУ2
[16876492] END | Заявка#3 | ОУ2
[16876497] START | Заявка#3 | ОУ3
[16877377] END | Заявка#3 | ОУ3
\end{lstlisting}

\section*{Вывод}

В данной части были проведены исследования производительности конвейерной архитектуры
для линейного и конвейерного вариантов алгоритма обработки заявок. Результаты показывают,
что конвейерная обработка обеспечивает сокращение времени выполнения по сравнению
с линейной для всех исследованных значений \(N\).

Анализ логов подтверждает параллельную работу обслуживающих устройств
и конвейерный характер обработки: несколько заявок одновременно находятся на разных стадиях
обслуживания.

\clearpage
