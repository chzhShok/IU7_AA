\chapter{Аналитическая часть}

\section{Конвейерная обработка данных}

\textbf{Конвейер}~\cite{conveyer,pipline_proc}~-- организация вычислений, при которой увеличивается количество выполняемых операций за единицу времени за счёт использования принципов параллельности. Конвейеризация в общем случае основана на разделении подлежащей исполнению функции на более мелкие части, называемые ступенями, и выделении для каждой из них отдельного блока аппаратуры или программного потока.

Обработка данных разбивается на несколько этапов, организуется передача результатов от одного этапа к следующему, и становится возможным совмещение выполнения разных этапов для различных данных. Производительность при этом возрастает за счёт того, что одновременно на различных ступенях конвейера обрабатываются разные элементы данных. При этом конвейерная обработка, как правило, не сокращает время обработки одной отдельной заявки, но увеличивает пропускную способность системы в целом.

\section{Структура конвейерной обработки заявок}

В качестве объекта конвейерной обработки в данной работе выступают заявки на поиск кратчайших путей в графе. Каждая заявка содержит:
\begin{itemize}
	\item имя файла с графом в формате DOT;
	\item имя стартовой вершины;
	\item список имён целевых вершин, для которых необходимо найти кратчайшие пути.
\end{itemize}

Для обработки набора таких заявок используется конвейер из трёх обслуживающих устройств (ОУ):
\begin{itemize}
	\item ОУ1 загружает граф из файла, находит индексы стартовой и целевых вершин во внутреннем представлении графа и дополняет заявку этой информацией;
	\item ОУ2 запускает параллельный алгоритм Дейкстры для соответствующего графа и стартовой вершины, формируя массив расстояний до всех вершин и массив предков для восстановления путей;
	\item ОУ3 по полученным расстояниям и массиву предков восстанавливает кратчайшие пути до целевых вершин, выбирает путь минимальной длины и формирует текстовый отчёт по заявке;
\end{itemize}

Между ступенями расположены блокирующие очереди заявок: если очередь пуста, поток, обслуживающий соответствующее ОУ, блокируется до появления новой заявки. Благодаря этому одновременно могут обрабатываться несколько заявок, находящихся на разных этапах конвейера, что повышает пропускную способность системы по сравнению с линейной обработкой заявок.

\section*{Вывод}

В аналитической части:
\begin{itemize}
	\item рассмотрено понятие конвейерной обработки данных и её влияние на пропускную способность системы;
	\item описана структура конвейера обработки заявок на запуск алгоритма Дейкстры.
\end{itemize}

\clearpage
