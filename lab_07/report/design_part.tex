\chapter{Конструкторская часть}

\section{Поток обработки}

Разрабатываемое программное обеспечение представляет собой консольное приложение, которое принимает на вход путь к PDF-файлу или каталогу и проверяет документы на наличие смешения типов нумерации иллюстрирующих элементов.

\begin{figure}[h]
	\centering
	\includegraphics[scale=0.5]{images/deepseek_1.png}
	\caption{Алгоритм программы, написанной Deepseek, часть 1}
	\label{fig:deepseek_1}
\end{figure}

\begin{figure}[h]
	\centering
	\includegraphics[scale=0.5]{images/deepseek_2.png}
	\caption{Алгоритм программы, написанной Deepseek, часть 2}
	\label{fig:deepseek_2}
\end{figure}

\clearpage

\begin{figure}[h]
	\centering
	\includegraphics[scale=0.33]{images/gemini.png}
	\caption{Алгоритм программы, написанной Gemini}
	\label{fig:gemini}
\end{figure}

\clearpage

\begin{figure}[h]
	\centering
	\includegraphics[scale=0.33]{images/gpt.png}
	\caption{Алгоритм программы, написанной GPT}
	\label{fig:gpt}
\end{figure}

\clearpage

\section{Процесс взаимодействия с LLM}

Для взаимодействия с моделями использовался единый подробный промпт, описывающий постановку задачи, требования к использованию \texttt{PyPDF2} и регулярных выражений, а также формат ожидаемого результата. Один и тот же промпт последовательно отправлялся нескольким LLM; в ответ на него каждая модель предлагала собственный, независимый вариант решения.

\section*{Вывод}

В конструкторской части были приведены схемы алгоритмов, написанных Deepseek, Gemini-3, GPT-5.1. Также был описан процесс взаимодействия с LLM.

\clearpage
