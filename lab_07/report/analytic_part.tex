\chapter{Аналитическая часть}

\section{Постановка задачи}

В соответствии с вариантом требуется проверить PDF-документ на наличие смешения типов нумерации иллюстрирующих элементов (рисунков и таблиц). Под \textbf{сквозной} нумерацией будем понимать последовательность номеров вида \(1, 2, 3, \ldots\), общую для всех элементов одного типа во всём документе. Под \textbf{пораздельной} (секционной) нумерацией будем понимать номера вида \(1.1, 1.2, \ldots, 1.N, 2.1, 2.2, \ldots\), в которых первая цифра соответствует номеру раздела, а вторая~--- порядковому номеру иллюстрации внутри раздела.

Документ считается корректным, если:
\begin{itemize}
	\item все иллюстрирующие элементы используют только сквозную нумерацию;
	\item либо все иллюстрирующие элементы используют только пораздельную нумерацию.
\end{itemize}
Во всех остальных случаях (одновременное присутствие обоих типов нумерации либо смешение внутри одного типа элементов) считается, что в документе присутствует ошибка оформления.

\section{Основные определения}

Под \textbf{иллюстрирующим элементом} будем понимать таблицу или рисунок, имеющие подпись в тексте документа. Подпись содержит ключевое слово (<<Таблица>>, <<Рисунок>>, их сокращения или англоязычные аналоги) и номер.

Будем считать, что корректные подписи имеют один из следующих шаблонов:
\begin{itemize}
	\item для таблиц: \texttt{Таблица 1}, \texttt{Таблица 2.3}, \texttt{табл.~1}, \texttt{Table 4}, \texttt{Table 2.1} и т.п.;
	\item для рисунков: \texttt{Рисунок 1}, \texttt{Рисунок 3.2}, \texttt{рис.~5}, \texttt{Figure 1}, \texttt{Fig. 2.4} и т.п.
\end{itemize}

\section{Регулярные выражения и конечные автоматы}

Поиск подписей иллюстрирующих элементов в тексте осуществляется с помощью регулярных выражений~\cite{re,regex-book}, которые могут быть сведены к детерминированным конечным автоматам. В решении, реализованном с помощью модели GPT~5.1, используется следующее обобщённое регулярное выражение:
\[
	\texttt{(рис.|рисунок|табл.|таблица|fig.|figure|table)\textbackslash s* \textbackslash d+(\textbackslash.\textbackslash d+)*}
\]
Выражение состоит из трёх логических частей:
\begin{itemize}
	\item группа ключевых слов для обозначения иллюстрирующих элементов на русском и английском языках;
	\item последовательность пробельных символов между ключевым словом и номером;
	\item номер, представляющий собой целое число (\(\texttt{\textbackslash d+}\)) c необязательной последовательностью точек и дополнительных чисел (\(\texttt{(\textbackslash.\textbackslash d+)*}\)).
\end{itemize}

После поиска совпадений по регулярному выражению номер анализируется отдельно: если он содержит точку, то считается пораздельной нумерацией, иначе~--- сквозной.

Аналогичный подход используется и в других реализациях, сгенерированных LLM (DeepSeek и Gemini), отличаясь лишь деталями шаблонов и используемых библиотек (\texttt{PyPDF2}) для извлечения текста~\cite{pypdf2}.

\section{Описание используемых LLM}

В рамках лабораторной работы были использованы три больших языковых модели для генерации и доработки программных решений~\cite{gpt,deepseek,gemini}:
\begin{itemize}
	\item \textbf{GPT~5.1}~--- облачная модель семейства ChatGPT, обладающая поддержкой развёрнутого пошагового рассуждения.
	\item \textbf{Gemini~3}~--- модель компании Google, ориентированная на решение широкого круга задач, включая анализ и генерацию кода. 
	\item \textbf{DeepSeek}~--- модель семейства coder, специально нацеленная на задачи программирования.
\end{itemize}

\section{Извлечение текста из PDF}

Извлечение текста из PDF-документов выполняется с помощью библиотеки \texttt{PyPDF2}~\cite{pypdf2}. Для каждой страницы вызывается метод \texttt{page.extract\_text()}, полученный текст разбивается на строки методом \texttt{splitlines()}, после чего на каждой строке выполняется поиск регулярного выражения.

Следует учитывать ограничения подобного подхода:
\begin{itemize}
	\item качество и структура извлечённого текста зависят от внутреннего устройства PDF-документа (наличия текстовых слоёв, способа разметки и т.п.);
	\item возможны ситуации, когда подпись к иллюстрации будет фрагментирована по нескольким строкам или содержать нестандартное форматирование, что усложняет поиск;
	\item таблицы и рисунки как графические объекты не анализируются напрямую~--- проверяются только их подписи, извлечённые как текст.
\end{itemize}

\section{Оценка трудоёмкости}

Пусть \(N\)~--- суммарное число символов во всех строках текста, извлечённого из PDF-документа. Процесс проверки включает следующие этапы:
\begin{itemize}
	\item проход по всем страницам документа и извлечение текста~--- линейно по размеру текста;
	\item разбиение текста на строки и последовательный просмотр каждой строки;
	\item применение регулярного выражения к каждой строке.
\end{itemize}

Регулярное выражение, используемое для поиска подписей, не содержит операторов, приводящих к экспоненциальному росту времени (например, вложенных квантификаторов с возвратами), и может быть реализовано как конечный автомат. Поэтому суммарная трудоёмкость алгоритма пропорциональна длине входных данных и оценивается как \(O(N)\).

Дополнительная память требуется только для хранения текущей строки, списка найденных совпадений и некоторого количества счётчиков, что даёт \(O(1)\) дополнительной памяти при потоковой обработке.

\section*{Вывод}

В аналитической части сформулирована постановка задачи проверки PDF-документов на наличие смешения типов нумерации иллюстрирующих элементов, введены основные определения и описаны используемые регулярные выражения. Рассмотрены особенности извлечения текста из PDF и показано, что трудоёмкость алгоритма линейна по количеству символов в документе при потоковой обработке.

\clearpage
