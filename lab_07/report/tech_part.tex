\chapter{Технологическая часть}

\section{Средства реализации}

Реализация программного обеспечения выполнена на языке Python~3.13. Для работы с PDF-документами и регулярными выражениями используются следующие библиотеки:
\begin{itemize}
	\item \texttt{PyPDF2}~--- основная библиотека для извлечения текста из PDF-файлов в итоговом решении~\cite{pypdf2};
	\item стандартный модуль \texttt{re} для работы с регулярными выражениями~\cite{re};
	\item дополнительные стандартные модули Python (\texttt{argparse}, \texttt{pathlib}, \texttt{typing}, \texttt{dataclasses}) для организации кода и типизации.
\end{itemize}

Разработка и тестирование выполнялись в виртуальном окружении \texttt{venv} на операционной системе macOS.

\section{Основные фрагменты реализации}

Ключевой фрагмент итоговой реализации функции поиска смешения типов нумерации, сгенерированной и доработанной на основе модели GPT~5.1, приведён в приложении~А (см. раздел~\ref{app:gpt51}).

Обёртка командной строки обеспечивает:
\begin{itemize}
	\item разбор аргументов командной строки (указание пути к файлу или каталогу);
	\item поиск всех PDF-файлов в указанной директории;
	\item поочередную обработку файлов и вывод краткого отчёта по каждому.
\end{itemize}

Реализация, полученная от модели DeepSeek, использует библиотеку \texttt{PyPDF2} и обобщённое регулярное выражение для поиска подписей таблиц и рисунков, определяет для каждой подписи тип нумерации (сквозная или пораздельная), подсчитывает распределение схем и считает нарушающими подписи с <<минорными>> схемами. Скрипт содержит собственную консольную обёртку для обработки отдельных файлов и каталогов, формирования сводной таблицы результатов и копирования проблемных документов в каталог \texttt{files\_with\_errors} (см. приложение~А, раздел~\ref{app:deepseek}).

Реализация, сгенерированная моделью Gemini-3, также основана на библиотеке \texttt{PyPDF2} и использует два регулярных выражения: для сквозной и для пораздельной нумерации. Функция \texttt{check\_mixed\_numbering} определяет факт смешения схем нумерации и возвращает список найденных подписей, а консольная часть последовательно обрабатывает набор файлов и выводит подробный человекочитаемый отчёт по каждому из них (см. приложение~А, раздел~\ref{app:gemini}).

\section{Функциональные тесты}

Для проверки корректности работы программы были сформулированы типовые тестовые сценарии, отражающие разные варианты оформления нумерации иллюстраций. На каждом сценарии выполнялся запуск итоговой реализации \texttt{search\_mixed\_numbering\_in\_pdf}. Сводка сценариев, ожидаемых и фактических результатов приведена в таблице~\ref{tbl:functional_tests}.

\begin{table}[h]
    \begin{center}
        \caption{\label{tbl:functional_tests}Результаты функциональных тестов}
        \begin{tabular}{|p{5cm}|p{5cm}|p{5cm}|}
            \hline
            \textbf{Тестовый сценарий} &
            \textbf{Ожидаемый результат} &
            \textbf{Фактический результат} \\ 
            \hline
            Только сквозная нумерация для всех рисунков и таблиц &
            Смешение не обнаружено &
            Смешение не обнаружено \\ 
            \hline
            Только пораздельная нумерация для всех иллюстраций &
            Смешение не обнаружено &
            Смешение не обнаружено \\ 
            \hline
            Рисунки с пораздельной нумерацией &
            Обнаружено смешение &
            Обнаружено смешение \\ 
            \hline
            Документ без иллюстрирующих элементов &
            Смешение не обнаружено &
            Смешение не обнаружено \\ 
            \hline
            Подписи только на английском языке &
            Корректное распознавание английских подписей &
            Корректное распознавание английских подписей \\ 
            \hline
        \end{tabular}
    \end{center}
\end{table}

\section{Промпты и результаты взаимодействия}

В ходе разработки были использованы три большие языковые модели (GPT~5.1, Gemini-3 и DeepSeek)~\cite{gpt,gemini,deepseek}, которым формулировались подробные текстовые задания на генерацию кода. Примеры промптов и фрагменты ответов моделей с сгенерированными реализациями приведены в приложении~Б: для DeepSeek (раздел~\ref{appB:deepseek_prompt}), для Gemini-3 (раздел~\ref{appB:gemini_prompt}) и для GPT~5.1 (раздел~\ref{appB:gpt_prompt}). Эти материалы иллюстрируют эволюцию решений и различия в подходах моделей к реализации одного и того же алгоритма.

\section*{Вывод}

В технологической части были выбраны и обоснованы средства реализации, приведены ключевые листинги программных модулей, сгенерированных с помощью больших языковых моделей, а также описаны результаты функционального тестирования на типовых сценариях. Итоговое программное обеспечение корректно выявляет случаи смешения типов нумерации иллюстрирующих элементов и удовлетворяет требованиям задания.
\clearpage
