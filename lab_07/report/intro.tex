\ssr{ВВЕДЕНИЕ}

Цель работы~--- разработать и реализовать программное обеспечение на языке Python для извлечения данных из PDF-документов с использованием библиотеки PyPDF2 и регулярных выражений.

Вариант задания: проверка наличия смешения типов нумерации для всех иллюстрирующих элементов документа: допустима либо только сквозная нумерация вида \(1, 2, 3, \ldots\), либо только пораздельная нумерация вида \(1.1, 1.2, \ldots, 1.N, 2.1, 2.2, \ldots\). Одновременное использование обоих подходов считается ошибкой.

Для достижения поставленной цели необходимо решить следующие задачи:
\begin{itemize}
	\item разработать регулярные выражения для поиска подписей иллюстрирующих элементов (рисунков и таблиц) в тексте PDF-документа и определения типа их нумерации;
	\item реализовать функцию для поиска в PDF-файле иллюстрирующих элементов и определения наличия смешения типов нумерации, возвращающую кортеж из логического признака и списка координат найденных элементов;
	\item реализовать консольное программное обеспечение, принимающее на вход путь к PDF-файлу или каталогу и использующее разработанную функцию;
	\item с помощью не менее трёх больших языковых моделей (LLM) получить различные реализации решения задачи, проанализировать их и при необходимости доработать;
	\item провести экспериментальную проверку работы программ на приложенном наборе тестовых PDF-файлов и сформировать таблицу с результатами;
	\item проанализировать качество решений, сгенерированных различными LLM.
\end{itemize}

\clearpage
