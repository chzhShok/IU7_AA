\ssr{ЗАКЛЮЧЕНИЕ}

В ходе выполнения лабораторной работы была достигнута поставленная цель~--- разработано и реализовано программное обеспечение на языке Python для извлечения текста из PDF-документов и проверки их на наличие смешения типов нумерации иллюстрирующих элементов.

Были решены следующие задачи:
\begin{itemize}
	\item сформулирована постановка задачи и введены понятия сквозной и пораздельной нумерации для иллюстраций;
	\item разработаны регулярные выражения для поиска подписей таблиц и рисунков на русском и английском языках;
	\item реализована функция поиска иллюстрирующих элементов в PDF-файле, возвращающая логический признак наличия смешения типов нумерации и список координат найденных элементов;
	\item реализовано консольное приложение, обрабатывающее одиночные файлы и каталоги с PDF-документами;
	\item с использованием трёх больших языковых моделей (DeepSeek, Gemini-3, GPT~5.1) получены и проанализированы различные варианты реализации;
	\item проведены функциональные тесты, подтверждающие корректность работы программы.
\end{itemize}

\clearpage


