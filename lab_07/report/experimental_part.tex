\chapter{Исследовательская часть}

\section{Характеристики ЭВМ}

Экспериментальные исследования проводились на ноутбуке со следующими характеристиками:
\begin{itemize}
	\item процессор: Apple M4 Pro;
	\item количество логических ядер: 12;
	\item оперативная память: 24~ГБ;
	\item операционная система: macOS Sequoia 15.6.1.
\end{itemize}

Во время измерений ноутбук был загружен только системными процессами и средой разработки.

\section{Набор данных}

Для проверки корректности реализованных алгоритмов использовался набор тестовых PDF-файлов из каталога \texttt{tests}. Основная последовательность файлов \texttt{00.pdf}~-- \texttt{13.pdf} описана в файле \texttt{tests\_description.md}. В нём указано, что:
\begin{itemize}
	\item файлы \texttt{00.pdf}~-- \texttt{12.pdf} не содержат ошибок оформления нумерации;
	\item файл \texttt{13.pdf} содержит ошибку (смешение типов нумерации иллюстрирующих элементов).
\end{itemize}

Дополнительно были использованы файлы с англоязычными подписями и смешанными сценариями нумерации (например, \texttt{test\_english\_...}, \texttt{test\_mixed\_...}), позволяющие проверить устойчивость регулярных выражений к различным языковым и форматным вариантам.

\section{Методика исследования}

Для каждого тестового файла выполнялись следующие шаги:
\begin{enumerate}
	\item запуск программы с указанием пути к файлу;
	\item получение результата работы функции:
	\begin{itemize}
		\item логический признак наличия смешения типов нумерации;
		\item список совпадений с координатами (страница, строка);
	\end{itemize}
	\item фиксация в таблице:
	\begin{itemize}
		\item названия PDF-файла;
		\item признака успешного/ошибочного оформления;
		\item координат первого найденного проблемного элемента (если он есть).
	\end{itemize}
\end{enumerate}

\section{Результаты тестов}

В таблице~\ref{tbl:tests_main} приведены результаты проверки для файлов \texttt{00.pdf}~-- \texttt{13.pdf}. Для файлов без ошибок программа возвращала \texttt{False} и пустой список, для файла с ошибкой~--- \texttt{True} и список с координатами.

\begin{table}[h]
	\begin{center}
		\caption{\label{tbl:tests_main}Результаты проверки тестовых PDF-файлов}
		\begin{tabular}{|p{5cm}|p{5cm}|p{5cm}|}
			\hline
			\textbf{Название файла} & \textbf{Смешение нумерации} & \textbf{Координаты первого нарушения} \\
			\hline
			00.pdf & нет  & --- \\
			\hline
			01.pdf & нет  & --- \\
			\hline
			02.pdf & нет  & --- \\
			\hline
			03.pdf & нет  & --- \\
			\hline
			04.pdf & нет  & --- \\
			\hline
			05.pdf & нет  & --- \\
			\hline
			06.pdf & нет  & --- \\
			\hline
			07.pdf & нет  & --- \\
			\hline
			08.pdf & нет  & --- \\
			\hline
			09.pdf & нет  & --- \\
			\hline
			10.pdf & нет  & --- \\
			\hline
			11.pdf & нет  & --- \\
			\hline
			12.pdf & нет  & --- \\
			\hline
			13.pdf & да   & стр. 49, линия 1; стр. 51, линия 7; стр. 52, линия 8 \\
			\hline
		\end{tabular}
	\end{center}
\end{table}

Помимо основной последовательности \texttt{00.pdf}~-- \texttt{13.pdf} были проверены дополнительные тестовые файлы из каталога \texttt{tests}, моделирующие типичные и ошибочные сценарии нумерации для русскоязычных и англоязычных документов. Результаты приведены в таблице~\ref{tbl:tests_additional}.

\begin{table}[h]
	\begin{center}
		\caption{\label{tbl:tests_additional}Результаты проверки дополнительных тестовых PDF-файлов}
		\begin{tabular}{|p{7cm}|p{4cm}|p{4cm}|}
			\hline
			\textbf{Название файла} & \textbf{Смешение нумерации} & \textbf{Координаты первого нарушения} \\
			\hline
			\texttt{test\_english\_through.pdf} & нет & --- \\
			\hline
			\texttt{test\_english\_sectional.pdf} & нет & --- \\
			\hline
			\texttt{test\_russian\_through.pdf} & нет & --- \\
			\hline
			\texttt{test\_russian\_sectional.pdf} & нет & --- \\
			\hline
			\texttt{test\_mixed\_names\_consistent.pdf} & нет & --- \\
			\hline
			\texttt{test\_english\_figures\_mixed.pdf} & да & стр. 1, линия 3 \\
			\hline
			\texttt{test\_russian\_tables\_mixed.pdf} & да & стр. 1, линия 3 \\
			\hline
			\texttt{test\_mixed\_between\_types.pdf} & да & стр. 1, линия 4; стр. 1, линия 5 \\
			\hline
			\texttt{test\_mixed\_numbering\_english.pdf} & да & стр. 1, линия 3; стр. 1, линия 4 \\
			\hline
			\texttt{test\_mixed\_numbering\_russian.pdf} & да & стр. 1, линия 3; стр. 1, линия 4 \\
			\hline
		\end{tabular}
	\end{center}
\end{table}

\vspace{300pt}
\section{Оценка качества решений LLM}

В ходе экспериментов сравнивались реализации, сгенерированные разными моделями:
\begin{itemize}
	\item решения DeepSeek и Gemini-3 в целом корректно обрабатывали большинство тестов, однако требовали доработки формата вывода и единообразия интерфейса;
	\item итоговая реализация GPT~5.1 показала наибольшую гибкость по отношению к различным формам подписей и удобный для отчёта формат результатов, поэтому была выбрана в качестве основной.
\end{itemize}

\section*{Вывод}

Экспериментальная проверка показала, что итоговая реализация корректно отличает документы без смешения типов нумерации от документов с ошибками оформления на заданном наборе тестовых файлов. Решение обладает достаточной устойчивостью к вариациям формата подписей и может использоваться для автоматизированной проверки отчётов и других технических документов.

\clearpage
