\ssr{ВВЕДЕНИЕ}

Целью данной работы является на материале графовых моделей алгоритмов выделить участки программ, которые могут быть исполнены параллельно.

Для достижения поставленной цели необходимо решить следующие задачи:

\begin{itemize}
	\item описать два алгоритма, согласно индивидуальному варианту, -- рекурсивный и нерекурсивный;
	\item описать реализации двух алгоритмов четырьмя графовыми моделями -- графом управления, информационным графом, операционной историей, информационной историей;
	\item указать участки каждой программы, которые могут быть исполнены параллельно, или отсутствие таковых.
\end{itemize}

\vspace{10pt}

В рамках данной работы рассматривается задача подсчёта количества единиц в последовательности из нулей и единиц, завершающейся числом 2. Анализируется последовательный код. В данной работе не требуется разрабатывать код, решающий задачу в параллельном режиме.

\clearpage