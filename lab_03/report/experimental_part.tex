\chapter{Исследовательская часть}

\section{Граф управление}
\subsection{Нерекурсивный алгоритм}
\begin{figure}[h!]\label{iterative_cfg} 
	\center{\includegraphics[scale=0.6]{./images/cfg_iterative.png}} 
	\caption{Граф управления нерекурсивного алгоритма}
\end{figure}

\subsection{Рекурсивный алгоритм}
\begin{figure}[h!]\label{recursive_cfg} 
	\center{\includegraphics[scale=0.6]{./images/cfg_recursive.png}} 
	\caption{Граф управления нерекурсивного алгоритма}
\end{figure} 

\section{Информационный граф}
\subsection{Нерекурсивный алгоритм}
\begin{figure}[h!]\label{info_graph_iterative} 
	\center{\includegraphics[scale=0.6]{./images/info_graph_iterative.png}} 
	\caption{Граф управления нерекурсивного алгоритма}
\end{figure}

\subsection{Рекурсивный алгоритм}
\begin{figure}[h!]\label{info_graph_recursive} 
	\center{\includegraphics[scale=0.6]{./images/info_graph_recursive.png}} 
	\caption{Граф управления нерекурсивного алгоритма}
\end{figure}

\section{Операционная история}
\subsection{Нерекурсивный алгоритм}
\begin{figure}[h!]\label{opt_history_iterative} 
	\center{\includegraphics[scale=0.5]{./images/opt_history_iterative.png}} 
	\caption{Граф управления нерекурсивного алгоритма}
\end{figure}

\subsection{Рекурсивный алгоритм}
\begin{figure}[h!]\label{opt_history_recurcive} 
	\center{\includegraphics[scale=0.5]{./images/opt_history_recurcive.png}} 
	\caption{Граф управления нерекурсивного алгоритма}
\end{figure}

\section{Информационная история}
\subsection{Нерекурсивный алгоритм}
\begin{figure}[h!]\label{info_history_iterative} 
	\center{\includegraphics[scale=0.5]{./images/info_history_iterative.png}} 
	\caption{Граф управления нерекурсивного алгоритма}
\end{figure}

\subsection{Рекурсивный алгоритм}
\begin{figure}[h!]\label{info_history_recurcive} 
	\center{\includegraphics[scale=0.5]{./images/info_history_recurcive.png}} 
	\caption{Граф управления нерекурсивного алгоритма}
\end{figure}

\section{Распараллеливание}
\subsection{Нерекурсивный алгоритм}
Нерекурсивный алгоритм не получится сделать параллельным, поскольку каждая итерация цикла зависит от результатов предыдущих: обновление переменной \texttt{count} требует знания её текущего значения, а условие досрочного завершения \texttt{break} создаёт управляющую зависимость, которая делает невозможным предсказание количества выполняемых итераций. Таким образом, все итерации образуют строго последовательную цепочку.

\subsection{Рекурсивный алгоритм}
Рекурсивный алгоритм не получится сделать параллельным, поскольку каждый рекурсивный вызов зависит от результата предыдущего: вычисление суммы требует полного завершения всех последующих рекурсивных вызовов. Таким образом, все вызовы образуют строго последовательную цепочку, где каждый шаг должен дождаться результата следующего перед тем как вернуть своё значение.

\section*{Вывод}

В данном разделе были построены граф управления, информационный граф,
операционная история и информационная история нерекурсивного и рекурсивного алгоритмов. Была проанализирована возможность параллельных вычислений для нерекурсивного и рекурсивного
алгоритмов.

\clearpage
