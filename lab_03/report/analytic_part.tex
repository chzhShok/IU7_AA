\chapter{Аналитическая часть}

\section{Рекурсия}

\subsection{Определение рекурсии}

\textbf{Рекурсия} -- это метод определения функций или решения задач, при котором функция вызывает саму себя непосредственно или через другие функции~\cite{goloveshin}. 

\vspace{10pt}

Рекурсивная функция $f$ определяется через~\cite{goloveshin,barron}:
\begin{itemize}
	\item \textbf{базовый случай} (терминальное условие): $f(x_0) = c$, где $x_0$ -- терминальное значение;
	\item \textbf{рекурсивный случай}: $f(x) = g(x, f(h(x)))$, где $h(x)$ -- функция, уменьшающая задачу.
\end{itemize}

\vspace{10pt}

Корректность рекурсивного алгоритма обеспечивается выполнением условий~\cite{goloveshin}:
\begin{itemize}
	\item \textbf{существование базового случая}: функция завершается при обнаружении терминатора 2 или пустой последовательности;
	\item \textbf{сходимость}: на каждом шаге размер задачи уменьшается ($|rest(S)| < |S|$);
	\item \textbf{правильность}: результат вычисляется как сумма вклада текущего элемента и результата для оставшейся последовательности.
\end{itemize}

\section{Графовые модели программ}

Программы можно представлять с помощью ориентированных графов, состоящих из набора вершин и множества соединяющих их направленных дуг. В зависимости от уровня детализации, вершины могут представлять различные сущности~\cite{fedotov}:

\begin{itemize}
	\item если вершины соответствуют итерациям циклов, то каждая вершина представляет операторы тела цикла, выполненные на одной и той же итерации;
	\item если вершины соответствуют срабатываниям операторов, то каждая вершина представляет один из операторов тела цикла, выполненный на некоторой итерации.
\end{itemize}

\subsection{Типы отношений в графовых моделях}

Дуги в графовых моделях программ отражают связь (отношение) между вершинами. Выделяют два основных типа отношений:

\begin{enumerate}
	\item \textbf{операционное отношение}: две вершины $A$ и $B$ соединяются направленной дугой тогда и только тогда, когда вершина $B$ может быть выполнена сразу после вершины $A$. Операционное отношение называют отношением по передаче управления;
	\item \textbf{информационное отношение}: две вершины $A$ и $B$ соединяются направленной дугой тогда и только тогда, когда вершина $B$ использует в качестве аргумента некоторое значение, полученное в вершине $A$. Информационное отношение называют отношением по передаче данных.
\end{enumerate}

\subsection{Основные модели программ}

На основе различных комбинаций типов вершин и отношений выделяют четыре основные графовые модели программ~\cite{fedotov}:

\begin{enumerate}
	\item \textbf{граф управления программы}: вершины являются операторами, а дуги представляют операционные отношения. Данная модель отражает поток управления в программе;
	\item \textbf{информационный граф программы}: вершины соответствуют операторам, а дуги представляют информационные отношения. Модель отражает зависимости по данным между операторами программы;
	\item \textbf{операционная история программы}: вершины представляют срабатывания операторов, а дуги отражают операционные отношения между ними;
	\item \textbf{информационная история программы}: вершины соответствуют срабатываниям операторов, а дуги представляют информационные отношения между ними.
\end{enumerate}

\section{Параллелизм в алгоритмах и программах}
Для определения ресурса параллелизма в графе алгоритма используется ярусно параллельная форма графа алгоритма~\cite{aho}. 

\textbf{Свойства}:
\begin{enumerate}
	\item начальная вершина каждой дуги расположена на ярусе с номером меньшим, чем номер яруса конечной вершины;
	\item между вершинами, расположенными на одном ярусе, не может быть дуг.
\end{enumerate}

\subsection{Закон Амдала}
\textbf{Закон Амдала} -- иллюстрирует ограничение роста производительности вычислительной системы с увеличением количества вычислителей~\cite{amdahl}.

$$\frac{T^1}{T^p} = S < \frac{1}{f + \frac{1 - f}{p}}$$

где
\begin{itemize}
	\item f -- доля последовательных операций $(0 \le f \le 1)$;
	\item p -- число процессоров;
	\item $T^1$ -- время работы программы на одном процессоре;
	\item $T^p$ -- время работы программы на системе из p процессоров.
\end{itemize}

\section*{Вывод}

В аналитической части:
\begin{itemize}
	\item представлено строгое математическое описание рекурсивного метода с выделением базового и рекурсивного случаев;
	\item рассмотрены графовые модели программ, включая четыре основных типа: граф управления, информационный граф, операционная история и информационная история программы;
	\item представлено определение ресурса параллелизма в графе алгоритма, используя ярусно параллельную форму графа алгоритма.
\end{itemize}

\clearpage