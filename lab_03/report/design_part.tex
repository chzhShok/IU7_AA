\chapter{Конструкторская часть}

\section{Требования к реализации}

К программе предъявлены ряд функциональных требований: входные данные -- последовательность целых чисел (0, 1, 2), выходные данные -- количество единиц в последовательности до первого вхождения числа 2.

\vspace{10pt}

К программе предъявлены ряд требований:
\begin{itemize}[label=---]
	\item наличие интерфейса для выбора действий;
	\item реализация рекурсивного алгоритма подсчёта единиц в последовательности;
	\item реализация нерекурсивного алгоритма подсчёта единиц в последовательности;
	\item наличие функциональности замера процессорного времени выполнения алгоритмов;
	\item поддержка ввода последовательности чисел с проверкой корректности данных.
\end{itemize}

\vspace{10pt}

Программа работает в трёх режимах: 
\begin{itemize}[label=---]
	\item выполнение рекурсивного алгоритма;
	\item выполнение нерекурсивного алгоритма;
	\item замер процессорного времени выполнения алгоритмов.
\end{itemize}

При первом и втором режимах на вход поступает последовательность чисел (0, 1), завершающаяся числом 2, на выход -- количество единиц в последовательности до первого вхождения числа 2.

При третьем режиме работы на вход поступают минимальный размер последовательности, максимальный размер, шаг изменения размера и число итераций k (k $\ge$ 100), на выход -- результаты замера времени в виде таблицы, где каждая точка получается делением времени выполнения k идентичных расчётов на k.

\section {Разработка алгоритмов}

Раздел содержит блок-схемы, описывающие следующие алгоритмы: рекурсивный алгоритм подсчёта единиц в последовательности на рисунке~\ref{fig:iterative}, нерекурсивный алгоритм подсчёта единиц в последовательности на рисунке~\ref{fig:recursive}.

\begin{figure}[h!]
	\center{\includegraphics[scale=0.9]{./images/iterative.png}} 
	\caption{Схема нерекурсивного алгоритма}
	\label{fig:iterative} 
\end{figure}\clearpage

\begin{figure}[h!]
	\center{\includegraphics[scale=0.7]{./images/recursive.png}} 
	\caption{Схема рекурсивного алгоритма}
	\label{fig:recursive} 
\end{figure}\clearpage

\section*{Вывод}

В данном разделе были разработаны два алгоритма подсчёта единиц в последовательности: рекурсивный и нерекурсивный. Для каждого из них представлены блок-схемы, описывающие логику работы.

\clearpage
