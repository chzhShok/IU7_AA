\chapter{Технологическая часть}

\section{Средства реализации}

Для реализации алгоритмов подсчёта единиц в последовательности был выбран язык программирования \texttt{Rust}, поскольку он соответствует требованиям лабораторной работы. Измерение времени выполнения алгоритмов осуществлялось с использованием структуры \texttt{Instant} из стандартной библиотеки \texttt{std::time}. Разработка программного обеспечения проводилась в интегрированной среде разработки \texttt{RustRover}.

\section{Реализация алгоритмов}

Были реализованы нерекурсивный алгоритм подсчёта единиц в последовательности (листинг 3.1), рекурсивный алгоритм подсчёта единиц в последовательности (листинг 3.2).

\begin{lstlisting}[caption={Нерекурсивный алгоритм}, label={lst:iterative}]
pub(crate) fn count_until_2(slice: &[i32]) -> usize {
	let mut count = 0; // 1
	for &x in slice { // 2
		if x == 2 { // 3
			break; // 4
		}
		count += (x == 1) as usize; // 5
	}
	count // 6
}
\end{lstlisting}

\begin{lstlisting}[caption={Рекурсивный алгоритм}, label={lst:recursive}]
pub(crate) fn count_until_2(slice: &[i32]) -> usize {
	if slice[0] == 2 { // 1
		0; // 2
	}
	let this = slice[0] as usize; // 3
	this + count_until_2(&slice[1..]) // 4
}
\end{lstlisting}

\section*{Вывод}

В данном разделе были реализованы алгоритмы нерекурсивный алгоритм подсчёта единиц в последовательности и рекурсивный алгоритм подсчёта единиц в последовательности. 

\clearpage
