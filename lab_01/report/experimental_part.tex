\chapter{Исследовательская часть}

\section{Характеристики ЭВМ}
Замеры проводились на устройстве со следующими характеристиками:

\begin{itemize} 
	\item процессор Apple M4 Pro;
	\item оперативная память 24 Гб;
	\item операционная система macOS Sequoia 15.6.1.
\end{itemize}

Замеры времени проводились, когда ноутбук был загружен только системными приложениями.

\section{Результаты замера времени}

В данном разделе представлены результаты измерения времени выполнения алгоритмов умножения матриц. Замеры проводились для квадратных матриц размером от 100×100 до 800×800 с шагом 100, а также для матриц размером от 101×101 до 801×801 с шагом 100. Каждый замер выполнялся 100 раз для получения более точных результатов.

\begin{table}[H]
	\centering
	\caption{Результаты замера времени для матриц размером 100-800}
	\begin{tabular}{|c|c|c|c|}
		\hline
		\textbf{N} & \textbf{Классический (мкс)} & \textbf{Виноград (мкс)} & \textbf{Опт. Виноград (мкс)} \\
		\hline
		100 & 1713 & 1362 & 1357 \\
		\hline
		200 & 12684 & 10152 & 10159 \\
		\hline
		300 & 46280 & 37632 & 37420 \\
		\hline
		400 & 109879 & 90012 & 89321 \\
		\hline
		500 & 222837 & 179346 & 168155 \\
		\hline
		600 & 385627 & 314130 & 292214 \\
		\hline
		700 & 618165 & 503980 & 481834 \\
		\hline
		800 & 938280 & 766001 & 732577 \\
		\hline
	\end{tabular}
	\label{time_results_100_800}
\end{table}

На основе таблицы 4.1 был построен график зависимости времени от размера матриц
(рисунок 4.1).

\begin{figure}[H]\label{even} 
	\center{\includegraphics[scale=0.4]{./images/algorithms_comparison_even.png}} 
	\caption{Сравнение времени на чётных размерах}
\end{figure}


\begin{table}[H]
	\centering
	\caption{Результаты замера времени для матриц размером 101-801}
	\begin{tabular}{|c|c|c|c|}
		\hline
		\textbf{N} & \textbf{Классический (мкс)} & \textbf{Виноград (мкс)} & \textbf{Опт. Виноград (мкс)} \\
		\hline
		101 & 1774 & 1437 & 1423 \\
		\hline
		201 & 13426 & 10755 & 10730 \\
		\hline
		301 & 48758 & 39697 & 39373 \\
		\hline
		401 & 114676 & 94518 & 93645 \\
		\hline
		501 & 229696 & 186372 & 175773 \\
		\hline
		601 & 396448 & 322997 & 310996 \\
		\hline
		701 & 630847 & 513647 & 501286 \\
		\hline
		801 & 952198 & 775787 & 742479 \\
		\hline
	\end{tabular}
	\label{time_results_101_801}
\end{table}

На основе таблицы 4.2 был построен график зависимости времени от размера матриц
(рисунок 4.2 ).

\begin{figure}[H]\label{odd} 
	\center{\includegraphics[scale=0.4]{./images/algorithms_comparison_odd.png}} 
	\caption{Сравнение времени на нечётных размерах}
\end{figure}


\section*{Вывод}

Согласно результатам исследования, алгоритм Винограда работает быстрее стандартного метода умножения матриц. Эта разница в скорости становится особенно заметной с увеличением размеров матриц. Наилучшая производительность достигается оптимизированной версией алгоритма Винограда, которая улучшена за счёт двух ключевых изменений: предварительный расчёт переменных и замена умножения на два на двоичный сдвиг целого числа.

\clearpage
